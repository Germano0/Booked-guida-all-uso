\chapter{Introduzione}
La presente guida ha come oggetto l'utilizzo della piattaforma Booked scheduler
per gestione elettronica delle prenotazioni di aule dell'ateneo.
Per un'agevole comprensione del testo, vengono introdotti di seguito alcuni concetti preliminari.
Booked gestisce tre tipi fondamentali di oggetti:
\begin{itemize}
 \item calendari;
 \item risorse;
 \item prenotazioni.
\end{itemize}
\paragraph*{Calendari}\mbox{}\\ %per andare a capo dopo nome paragrafo.
I calendari presenti sono:
\begin{itemize}
 \item Default;
 \item Economia;
 \item Ingegneria;
 \item Lettere;
 \item Medicina;
 \item Scienze.
\end{itemize}

in maniera tale da separare le esigenze di prenotazione delle varie macro aree d'ateneo.
Il calendario ``Default'' è un calendario predefinito e vuoto, che appare quando si accede
alla pagina web dei calendari, pertanto occorre scegliere quello di propria competenza.

\paragraph*{Risorse}\mbox{}\\ %per andare a capo dopo nome paragrafo.
Booked gestisce prenotazioni di risorse: esse possono essere laboratori, laboratori,
mezzi di trasporto, aule didattiche, ecc. Queste ultime pertanto, verranno viste dal
programma come delle risorse.


\paragraph*{Prenotazioni}\mbox{}\\ %per andare a capo dopo nome paragrafo.
Le prenotazioni vengono effettuate operando sulla pagina che mostra il calendario: si sceglie
la risorsa da prenotare e si impostano le proprie preferenze.