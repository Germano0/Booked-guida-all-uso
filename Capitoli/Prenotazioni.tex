\chapter{Prenotazioni}
Per effettuare una prenotazione, cliccare la voce ``Calendario'' nel menù in alto alla pagina web.
Selezionare il calendario di propria competenza
\begin{figure}[H]
\centering{}\includegraphics[scale=0.5]{Immagini/calendari_selezione.pdf}
\normalsize
\caption{}
\label{fig:calendari_selezione.pdf}
\end{figure}

e scorrere la pagina fino a trovare la data adatta alla prenotazione. Sulle righe
sono presenti tutte le risorse (aule) disponibili.
Per iniziare a creare una prenotazione cliccare su un quadratino qualsiasi nella fascia oraria
dell'aula desiderata. È possibile modificare i dettagli orari nella schermata
successiva.

Per fissare una prenotazione su più giorni della settimana, andare nella zona ``Ripeti'' come in Figura
\ref{fig:prenotazione_ripetizione_1.pdf}

\begin{figure}[H]
\centering{}\includegraphics[scale=0.5]{Immagini/prenotazione_ripetizione_1.pdf}
\normalsize
\caption{}
\label{fig:prenotazione_ripetizione_1.pdf}
\end{figure}

e cliccare su ``Settimanale''

\begin{figure}[H]
\centering{}\includegraphics[scale=0.5]{Immagini/prenotazione_ripetizione_2.pdf}
\normalsize
\caption{}
\label{fig:prenotazione_ripetizione_2.pdf}
\end{figure}


La scritta ``Note prenotazione'' significa ``Nome prenotazione''. Vi è un errore di traduzione
del software, che è stato corretto verrà corretto a breve\footnote{Nota 04 Ottobre 2016: la traduzione è stata
cambiata, rimane da aggiornare la fotografia nella guida}.
In tale campo va inserito il nome della materia insegnata.
Non dimenticare di aggiungere il corso di studi della materia e l'anno corrispondente.


\begin{figure}[H]
\centering{}\includegraphics[scale=0.5]{Immagini/prenotazione_attributi.pdf}
\normalsize
\caption{}
\label{fig:prenotazione_attributi.pdf}
\end{figure}

Un esempio può essere

\begin{figure}[H]
 \centering{} Ingegneria informatica II anno
\normalsize
\end{figure}

Nel caso la prenotazione riguardi più corsi di studi, metterne uno per riga