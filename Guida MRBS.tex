%\documentclass[italian]{scrreprt}
\documentclass[italian,a4paper,12pt,oneside]{book}
\usepackage[T1]{fontenc}
\usepackage[latin9]{inputenc}
\usepackage{geometry}
\usepackage{graphicx}
\usepackage{forest}
\setcounter{secnumdepth}{3}
\setcounter{tocdepth}{1}
\usepackage{float}
\usepackage{babel}
\usepackage{microtype} % migliora espansione dei font. Suggerito su ``L'arte di scrivere in LaTeX, pag. 45''
\usepackage{indentfirst} % indentazione anche su primo paragrafo.  Suggerito su ``L'arte di scrivere in LaTeX, pag. 45''
\usepackage{booktabs} % serve per le tabelle.   Suggerito su ``L'arte di scrivere in LaTeX, pag. 85''
\usepackage{caption} % serve per le tabelle.   Suggerito su ``L'arte di scrivere in LaTeX, pag. 85''



\geometry{textheight=24.2cm,textwidth=16cm}
\pagestyle{headings}
\frenchspacing

\date{}

\title{
\huge Guida all'uso della piattaforma di prenotazione sale
\\ MRBS (Meeting Room Booking System)
} 
\author{Versione 0.01 \\
Germano Massullo}

\makeindex

\begin{document}
\frontmatter		% inizia la numerazione con numeri romani

\maketitle
\tableofcontents

\chapter*{Prefazione}
La presente guida tratta l'utilizzo della piattaforma MRBS per
le necessit� dell'ateneo


\mainmatter		% inizia la numerazione con numeri cardinali
\chapter{Operazioni post installazione}
\section{Inizializzazione aree e sale}
Dopo l'installazione della piattaforma MRBS, occorre popolare il sistema
con le informazioni che rispecchiano le esigenze dell'ateneo.
In MRBS ogni \emph{sala} appartiene ad una determinata \emph{area}.
Pertanto, il primo passo � creare tutte le aree di interesse, che possono
essere assimilate alle varie macro-aree dell'universit� (esempio:
ingegneria, lettere, scienze, ecc.).

\chapter{Prenotazioni e gestione credenziali utenti}



\end{document}
